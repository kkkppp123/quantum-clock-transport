% docs/theory/02_ptm_trace_reconstruction.tex
% v1.0 (PR-1)
\documentclass[11pt]{article}
\usepackage[a4paper,margin=2.6cm]{geometry}
\usepackage{amsmath,amssymb,amsthm}
\usepackage{mathtools}
\usepackage{bm}

\newtheorem{theorem}{Twierdzenie}
\newtheorem{lemma}{Lemat}
\newtheorem{remark}{Uwaga}

\title{Rekonstrukcja $|\mathrm{Tr}\,U|$ z przekątnej PTM dla jednego kubitu}
\author{quantum-clock-transport — Etap H / PR-1}
\date{}

\begin{document}
\maketitle

\paragraph{PTM i reprezentacja adjoint.}
Dla $U\in SU(2)$ niech $R\in SO(3)$ będzie macierzą transferu Pauliego (PTM),
zdefiniowaną przez $R_{ij}=\frac{1}{2}\mathrm{Tr}(\sigma_i\,U\,\sigma_j\,U^\dagger)$,
$i,j\in\{x,y,z\}$. Reprezentacja adjoint $SU(2)\to SO(3)$ daje identyfikację działania $U$
na wektorze Blocha przez obrót $R$.

\begin{theorem}[PTM $\rightarrow$ $|\mathrm{Tr}\,U|$]\label{thm:ptm-trace}
Niech $U\in SU(2)$ i niech $r_{xx},r_{yy},r_{zz}$ będą elementami przekątnymi odpowiadającej PTM $R$.
Wtedy zachodzi
\begin{equation}
|\mathrm{Tr}\,U|^2 \;=\; 1 + (r_{xx}+r_{yy}+r_{zz}).
\end{equation}
\end{theorem}

\begin{proof}[Szkic dowodu]
Parametryzuj $U=\exp(-\tfrac{i}{2}\theta\,\bm{n}\cdot\bm{\sigma})$, gdzie $\bm{n}$ jest jednostkowe.
Wtedy $\mathrm{Tr}\,U=2\cos(\theta/2)$, zatem $|\mathrm{Tr}\,U|^2=4\cos^2(\theta/2)=2+2\cos\theta$.
Z drugiej strony, odpowiadający obrót Blocha ma ślad $\mathrm{Tr}\,R=1+2\cos\theta$.
Sumując przekątne otrzymujemy $\mathrm{Tr}\,R=r_{xx}+r_{yy}+r_{zz}=1+2\cos\theta$,
a więc $|\mathrm{Tr}\,U|^2=1+\mathrm{Tr}\,R=1+(r_{xx}+r_{yy}+r_{zz})$.
\end{proof}

\begin{remark}[Interpretacja pomiarowa]
$r_{ii}$ to wartość oczekiwana pomiaru $\sigma_i$ po zadziałaniu $U$ na stan własny $+1$ operatora $\sigma_i$
(wejście i pomiar w tej samej bazie). W praktyce daje to trzy proste eksperymenty/symulacje
do odtworzenia $|\mathrm{Tr}\,U|$ bez wrażliwości na fazę globalną.
\end{remark}

\paragraph{Aspekt numeryczny.}
W implementacji (moduł \texttt{qct.holonomy.ptm}) liczymy $R_{ij}$ z definicji,
a następnie zwracamy przekątną $(r_{xx},r_{yy},r_{zz})$.
Formuła z Tw.~\ref{thm:ptm-trace} daje
\[
|\mathrm{Tr}\,U| \;=\; \sqrt{\,\max\{0,\; 1 + r_{xx}+r_{yy}+r_{zz}\}\,},
\]
co jest stabilne numerycznie (obcięcie do zera usuwa ujemności z zaokrągleń).

\end{document}
