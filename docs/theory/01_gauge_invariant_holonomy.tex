% docs/theory/01_gauge_invariant_holonomy.tex
% v1.0 (PR-1)
\documentclass[11pt]{article}
\usepackage[a4paper,margin=2.6cm]{geometry}
\usepackage{amsmath,amssymb,amsthm}
\usepackage{mathtools}
\usepackage{bm}

\newtheorem{theorem}{Twierdzenie}
\newtheorem{lemma}{Lemat}
\newtheorem{corollary}{Wniosek}
\newtheorem{remark}{Uwaga}

\title{Gauge-invariant holonomia: fazowo zminimalizowana odległość Frobeniusa}
\author{quantum-clock-transport — Etap H / PR-1}
\date{}

\begin{document}
\maketitle

\paragraph{Cel.}
Dla $U\in U(d)$ chcemy wyrazić \emph{niezależną od fazy globalnej} odległość od identyczności.
Pokażemy formułę zamkniętą na
\[
\min_{\phi\in\mathbb{R}}\;\bigl\|e^{i\phi}U-I\bigr\|_F.
\]

\begin{theorem}[Phase-minimized distance]\label{thm:phase-min}
Niech $U\in U(d)$. Wówczas
\begin{equation}
\min_{\phi\in\mathbb{R}}\;\bigl\|e^{i\phi}U-I\bigr\|_F
=\sqrt{\,2d-2\,|\mathrm{Tr}\,U|\,}.
\end{equation}
\end{theorem}

\begin{proof}[Szkic dowodu]
Dla dowolnego $\phi$:
\[
\|e^{i\phi}U-I\|_F^2
=\mathrm{Tr}\bigl[(e^{i\phi}U-I)^\dagger(e^{i\phi}U-I)\bigr]
=2d- e^{-i\phi}\mathrm{Tr}\,U - e^{i\phi}\overline{\mathrm{Tr}\,U}.
\]
Niech $z:=\mathrm{Tr}\,U\in\mathbb{C}$. Wtedy powyżej mamy $2d - 2\mathrm{Re}(e^{-i\phi}z)$.
Maksymalizacja części rzeczywistej po $\phi$ daje $\max_\phi \mathrm{Re}(e^{-i\phi}z)=|z|$,
co minimalizuje normę do wartości $2d-2|z|$.
Po spierwiastkowaniu otrzymujemy tezę.
\end{proof}

\begin{corollary}[Qubit ($d=2$)]
Dla $U\in U(2)$:
\[
\min_{\phi}\|e^{i\phi}U-I\|_F=\sqrt{\,4-2\,|\mathrm{Tr}\,U|\,}.
\]
\end{corollary}

\begin{remark}[Aspekt numeryczny]
W praktyce korzystamy bezpośrednio z $|\mathrm{Tr}\,U|$;
dla $d=2$ implementacja sprowadza się do \texttt{sqrt(max(0, 4-2*abs\_tr))}. 
Jeżeli $U\in SU(2)$, to $|\mathrm{Tr}\,U|\in[0,2]$.
\end{remark}

\paragraph{Zastosowanie (holonomia małej pętli).}
Przy ekspansji małej pętli $\Gamma$ (pole $\delta A$) mamy
$U_\Gamma=I - i\,\delta A\,F_{op}+O(\delta A^2)$,
więc $\|U_\Gamma-I\|_F = \delta A\,\|F_{op}\|_F + O(\delta A^2)$.
Odległość z Tw.~\ref{thm:phase-min} usuwa arbitralną fazę globalną, co czyni metrykę \emph{gauge-invariant}.

\end{document}
